%! Author = Adham Aijou
%! Date = 20/02/2023
% Preamble
\documentclass[11pt]{article}
\usepackage[ngerman]{babel}
\usepackage{hyperref}
\usepackage{wrapfig}
\usepackage{mdframed}
\usepackage[T1]{fontenc}
\usepackage[utf8]{inputenc}
\usepackage[paper=a4paper,left=40mm,right=20mm,top=25mm,bottom=20mm,footskip=25pt ]{geometry}
\usepackage[backend=biber,style=alphabetic, uniquename=false]{biblatex}
\usepackage{mathptmx} %Font Times New Roman
\usepackage[acronyms, toc,section,numberedsection = autolabel,automake,]{glossaries} %Erstellt ein Abkürzungsverzeichnis
\usepackage{array}
\usepackage{float}
\usepackage{python}
\usepackage{minted}
\usepackage{xcolor}
\usepackage{filecontents}
\usepackage{sectsty}
\usepackage{tocvsec2}
\usepackage[title,titletoc]{appendix}
\usepackage{enumitem}
\usepackage{pgfplots}
\usepackage{tikz}
\usepackage{longtable} % Tabellen können Seitenumbrüche enthalten
\usepackage{listings} % Listings für Codebeispiele
\renewcommand{\lstlistingname}{Codeausschnitt}
\usepackage{pdfpages} % PDFs einbinden
\usepackage{placeins} % FloatBarrier
\usepackage{multirow}
\usepackage{csquotes}
\usepackage{makecell}
\renewcommand{\sectionautorefname}{Abschnitt}
\renewcommand{\subsectionautorefname}{Unterabschnitt}
\renewcommand{\figureautorefname}{Abbildung}
\renewcommand{\tableautorefname}{Tabelle}
\usepackage{color}
%\addtokomafont{disposition}{\rmfamily} %Serifenschrift für Überschriften
\counterwithout{footnote}{section} %Durchgehende Fußnoten Nummerierung
\setlength\parindent{0pt} % Einrückung für neue Absätze
\usepackage[font=footnotesize,justification = centering]{caption} %Font für Bildunterschriften
\usepackage{pifont}   % Wird für den Punkt neben den Seitzenzahlen benötigt
\newcommand{\cmark}{\ding{51}}%
\newcommand{\xmark}{\ding{55}}%
\usepackage[footsepline=1pt]{scrlayer-scrpage} % Anpassung von Kopf/Fußzeilen
\usepackage{setspace}
\usepackage{microtype}
\usepackage{tabularx}
\usepackage{booktabs}
\usepackage{ragged2e}
\usepackage{amsfonts}
% Document
\patchcmd{\listoffigures}{\chapter*}{\chapter}{}{}
\makeglossaries
\defglsentryfmt{\color{black}\bfseries\glsgenentryfmt}
%\addbibresource{main.bib}
\addbibresource{literatur.bib}
\begin{document}
%    - Vorname vom Betreuer
%- Einleitung sollte Titel beinhalten
%- two footnotes on page 17 (Reflow)
%- Fragestellungen und Ziele nicht aufzählen, sondern schlussfolgern aus Problemstellung
%- Nicht begründen warum
%- Nicht Ziele/Fragen zuerst, erst Hinführung mit der Zusammenfassung
%- caniuse Glossar (Deutsch bitte)
%- 500ms Satz sollte dorthin, wo der Leser davon weiß
%- Die detaillierten Aspekte
%- Struktur der Arbeit nur in ein Kapitel
%- Tabellen maybe als Screenshot einfügen
    %! Author = Adham Aijou
%! Date = 20/02/2023

% Preamble



\begin{titlepage}
    \centering
    \includegraphics[width=0.35\textwidth]{Bilder/Fachhochschule_für_die_Wirtschaft_logo}\par\vspace{1cm}
    Fachhochschule für die Wirtschaft Hannover \\
    - FHDW -\\
    {\scshape\Large Bachelorarbeit \par}
    \vspace{1cm}
    {\huge\bfseries Konzeptionierung und Realisierung einer webbasierten Schnittstelle zur gerichteten Anforderungsabwicklung für Infor LN \par}
    \vspace{1cm}
    \normalsize{
        \begin{tabular}{l l}
            \textbf{Verfasser:} & \textbf{Adham Aijou} \\
            &	Brinker Straße 72 \\
            &	30851, Langenhagen \\
            \\

            \textbf{Erstbetreuer:} & \textbf{Prof. Dr. Ing. Volkhard Klinger}\\
            \\
            \textbf{Zweitbetreuer:} & \textbf{Prof. Dr. Günther Hellberg}\\
            \\
            \textbf{Studiengruppe:} & \textbf{HFI421IN}\\
            \\
            \textbf{Matrikelnummer:}  & \textbf{600142} \\
            \\
            \textbf{Ausbildungsbetrieb:}  & \textbf{ACTOP GmbH}\\
            & Heinrichstraße 27\\
            & 31303, Burgdorf    \\
            \\
            \textbf{Eingereicht am:} & 17.09.2024
        \end{tabular}\\
    }
    \vfill
    \flushright \includegraphics[width=0.2\textwidth]{Bilder/ACTOP GmbH Logo (NEW)}\par\vspace{1cm}

\end{titlepage}
% Document


    \pagenumbering{roman}

    %! Author = AdhamAijou
%! Date = 16/09/2024


%no page numbering

    \section*{Sperrvermerk}\label{sec:sperrvermerk}
    Diese Arbeit enthält vertrauliche Informationen des Unternehmens ACTOP GmbH.
    Dies betrifft alle Inhalte der Arbeit, die nicht aus öffentlichen Quellen stammen, sowie Codebeispiele und Konzepte, die spezifisch für das Unternehmen entwickelt wurden.
    Teile dieser Arbeit dürfen ohne ausdrückliche Genehmigung des Unternehmens nicht weitergegeben werden.
    Die Weitergabe und Vervielfältigung dieser Arbeit oder von Teilen daraus sind im Ganzen oder in Teilen untersagt.
    Ausnahmen bedürfen der schriftlichen Genehmigung des Unternehmens.
    Die Arbeit ist nur den Korrektoren und den Mitgliedern des Prüfungsausschusses zugänglich zu machen.
\newpage



%    \input{Zusammenfassung}
%    \newpage
    \input{Abstract}
    \input{Inhaltsverzeichnis}
    \pagebreak
\section{Einleitung}
Unternehmen verfügen über eine Fülle von Dokumenten und Daten, die auf Plattformen wie Microsoft SharePoint, OneDrive oder in Datenbanken abgelegt sind.
Mitarbeiter stehen oft vor der Herausforderung, benötigte Informationen aus dieser Menge an unstrukturiertem Wissen schnell zu extrahieren.
KI-basierte Chatbots versprechen hier Abhilfe, indem sie, anhand eines trainierten Datenmodells, in natürlicher Sprache Fragen beantworten können.
Allerdings sind viele frei verfügbare Chatbot-Lösungen (etwa große Sprachmodelle à la ChatGPT) für den direkten Einsatz auf interne Firmendaten nicht geeignet, da die Eingabedaten an externe Server übertragen werden und somit Datenschutzrisiken entstehen\href{https://punctuations.ai/ai-agents-workflows/your-private-gpt-the-case-for-secure-on-premise-llms/#:~:text=pharmaceutical%20companies%2C%20manufacturers%2C%20government%20contractors,keeping%20AI%20close%20to%20home}{punctuations.ai}.
Die Entwicklung eines \textit{datenschutzkonformen Chatbots} erfordert daher einen Ansatz, bei dem vertrauliche Dokumente im eigenen Verantwortungsbereich verbleiben (On-Premises-Betrieb) und dennoch die Leistungsfähigkeit moderner KI genutzt wird.


Diese Masterarbeit untersucht die Entwicklung eines RAG-gestützten Chatbots (Retrieval-Augmented Generation), der es ermöglicht, interne Dokumente automatisiert auszulesen und zur Beantwortung von Benutzeranfragen heranzuziehen.
Im Fokus stehen dabei Datenschutz und Machbarkeit: Das System soll ausschließlich innerhalb der unternehmenseigenen IT-Infrastruktur operieren, sodass keine sensiblen Inhalte nach außen gelangen.
Gleichzeitig soll der Chatbot die Vorteile von \textit{Large Language Models} (LLMs) nutzen, um den Nutzern präzise und kontextrelevante Antworten in einem Chat-Interface bereitzustellen.
Dokumente aus SharePoint, OneDrive und optional weiteren Quellen (z.B.\ firmeneigene Datenbanken) werden dafür als Wissensbasis eingebunden.


Die Motivation für diese Arbeit ergibt sich aus dem Bedarf, internes Wissen schneller zugänglich zu machen, ohne gegen Datenschutzrichtlinien zu verstoßen.
Im Rahmen der Arbeit werden zunächst die Problemstellung und Anforderungen analysiert, gefolgt von theoretischen Grundlagen zu LLMs, RAG und Datenschutz.
Darauf aufbauend werden die technische Umsetzung und das Vorgehen bei der Implementierung beschrieben.
Abschließend werden die erzielten Ergebnisse präsentiert und in der Zusammenfassung Schlussfolgerungen sowie ein Ausblick gegeben.

Es ergeben sich folgende Forschungsfragen, die im Verlauf der Arbeit beantwortet werden sollen:

\begin{enumerate}
    \item Wie kann ein RAG-Chatbot für firmeninterne Wissensdatenbanken technisch umgesetzt werden?

    \item Welche Mehrwerte bietet ein RAG-gestützter Chatbot im Vergleich zu klassischen Suchsystemen?

    \item Welche Faktoren beeinflussen die Qualität und Akzeptanz eines RAG-Chatbots im Unternehmenskontext?

    \item Welche Herausforderungen ergeben sich bei Datenschutz, Sicherheit und Integration in bestehende Systeme?

\end{enumerate}


Daraus leiten sich die Ziele der Arbeit ab:

\begin{itemize}
    \item Entwicklung eines Prototyps für einen RAG-Chatbot, der interne Dokumente aus SharePoint und OneDrive nutzt.
    \item Evaluation der Antwortqualität und Benutzerfreundlichkeit des Chatbots.
    \item Analyse der Datenschutzaspekte und Implementierung geeigneter Sicherheitsmaßnahmen.
    \item Identifikation von Herausforderungen und Best Practices für den Einsatz von RAG-Chatbots in Unternehmen.
    \item Wie ist der Entwicklungsaufwand für ein Kleinunternehmen ohne externe Cloud-Dienste realisierbar?
\end{itemize}


    \pagebreak
\section{Grundlagen}
\subsection{LLMs}
Large Language Models (LLMs) bilden die Basis moderner Chatbots.
Hierbei handelt es sich um KI-Sprachmodelle, die mit sehr großen Textmengen vortrainiert wurden (z.B. GPT-3/4/5, BERT, LLaMA).
Sie können kontextabhängig natürlichsprachliche Antworten erzeugen und Fragen in eigenen Worten beantworten.
LLMs haben jedoch inhärente Limitationen im Unternehmenskontext: Ohne zusätzliche Anbindung verfügen sie nur über allgemeines Weltwissen bis zu ihrem Trainingsstand, aber nicht über aktuelle oder firmeninterne Informationen.
Daraus resultiert die Gefahr, dass ein LLM auf spezifische Fragen entweder gar keine oder fehlerhafte Antworten (sog. \textit{Halluzinationen}) gibt\href{https://en.wikipedia.org/wiki/Retrieval-augmented_generation#:~:text=RAG%20improves%20large%20language%20models,4}{en.wikipedia.org}.
Beispielsweise kann ein Standard-LLM nicht wissen, welche internen Richtlinien ein bestimmtes Unternehmen hat, wenn diese Informationen nicht öffentlich sind oder im Training nicht berücksichtigt wurden.

\subsection{RAG}

Retrieval-Augmented Generation (RAG) ist ein KI-Architekturansatz, der genau diese Lücke schließt.
RAG kombiniert ein generatives Sprachmodell mit einem vorgeschalteten Information-Retrieval-Schritt\href{https://learn.microsoft.com/en-us/microsoftteams/platform/toolkit/build-a-rag-bot-in-teams#:~:text=The%20advanced%20Q%26A%20chatbots%20are,architecture%20has%20two%20main%20flows}{learn.microsoft.com}.
Das bedeutet: Bevor das LLM eine Antwort generiert, werden aus einer definierten Wissensbasis (z.B. dem Dokumentenbestand des Unternehmens) jene Informationen abgerufen, die für die gestellte Frage relevant sind.
Technisch wird dies oft so umgesetzt, dass die Benutzerfrage zunächst in einen semantischen Vektor umgewandelt wird, der mit zuvor indexierten Dokument-Vektoren verglichen wird.
Die passendsten Dokumentenabschnitte werden ermittelt und dem Prompt des LLM als zusätzlicher Kontext hinzugefügt\href{https://en.wikipedia.org/wiki/Retrieval-augmented_generation#:~:text=Retrieval,responses%20based%20on%20authoritative%20sources}{en.wikipedia.org}.
Erst im Anschluss formuliert das LLM auf Basis der angereicherten Eingabe die Antwort.
Die RAG-Architektur umfasst dabei typischerweise zwei Hauptphasen: \textit{(1) Offline-Datenaufbereitung} und \textit{(2) Online-Antwortgenerierung)}\href{https://learn.microsoft.com/en-us/microsoftteams/platform/toolkit/build-a-rag-bot-in-teams#:~:text=The%20advanced%20Q%26A%20chatbots%20are,architecture%20has%20two%20main%20flows}{learn.microsoft.com}.
In Phase (1) werden die Dokumente aus den Datenquellen eingespeist, vorverarbeitet (Textextraktion, Bereinigung, Chunking) und indexiert – etwa in Form von Vektorembeddings, die in einer Datenbank abgelegt werden.
Phase (2) wird bei jeder Nutzeranfrage durchlaufen: Das System sucht im Index nach den inhaltlich ähnlichsten Einträgen zur Frage und übergibt diese zusammen mit der Frage an das LLM, das daraus seine Antwort konstruiert.
Durch dieses Vorgehen kann das Sprachmodell aktuelle, spezifische Informationen verwenden, die nicht in seinem ursprünglichen Training vorhanden waren, etwa interne Dokumentinhalte\href{https://en.wikipedia.org/wiki/Retrieval-augmented_generation#:~:text=Retrieval,responses%20based%20on%20authoritative%20sources}{en.wikipedia.org}.
RAG erhöht somit die fachliche Korrektheit und Relevanz der Antworten: Das Modell „halluziniert“ weniger, da es an konkrete Fakten aus den Dokumenten gebunden wird\href{https://en.wikipedia.org/wiki/Retrieval-augmented_generation#:~:text=RAG%20improves%20large%20language%20models,4}{en.wikipedia.org}.
Zudem entfällt die Notwendigkeit, das LLM bei jeder Wissensänderung komplett neu zu trainieren – neue Informationen können einfach durch Aktualisierung der Wissensbasis einbezogen werden\href{https://en.wikipedia.org/wiki/Retrieval-augmented_generation#:~:text=RAG%20also%20reduces%20the%20need,to%20ensure%20accuracy%20and%20relevance}{en.wikipedia.org}, was erheblich effizienter ist.



\subsection{Datenschutz und IT-Sicherheit}

Neben den KI-Aspekten sind die Grundlagen des Datenschutzes und der IT-Sicherheit in diesem Kontext wesentlich.
In Europa setzt etwa die DSGVO strikte Grenzen, wie personenbezogene Daten verarbeitet und weitergegeben werden dürfen.
Übertragen auf unseren Anwendungsfall bedeutet dies: Die Inhalte interner Dokumente, die eventuell sensible Informationen enthalten, dürfen keinesfalls unkontrolliert nach außen gelangen.
Ein zentrales Prinzip ist daher, dass die Verarbeitung der Daten on-premises, also auf unternehmenseigener Infrastruktur, erfolgt.
Damit verbleiben alle Daten unter der Kontrolle der Organisation – ein entscheidender Vorteil gegenüber der Nutzung öffentlicher Cloud-LLM-Dienste\href{https://punctuations.ai/ai-agents-workflows/your-private-gpt-the-case-for-secure-on-premise-llms/#:~:text=1,premise%20deployment}{punctuations.ai}.
Bei einer On-Premises-Lösung kann garantiert werden, dass sämtliche Zwischenergebnisse (Extrakte der Dokumente, Vektor-Embeddings, Chat-Verläufe) das geschützte Netzwerk nicht verlassen.
Dies minimiert das Risiko von Datenlecks und erleichtert auch die Einhaltung von Compliance-Vorgaben, da genau nachvollzogen werden kann, wo und wie die Daten verarbeitet werden.


Ein weiterer Aspekt der Datenschutzkonformität betrifft die Zugriffskontrolle.
In SharePoint etwa ist nicht jedes Dokument für jeden Mitarbeiter einsehbar; es existieren Berechtigungskonzepte.
Ein wirklich produktiver, interner Chatbot müsste diese rollenbasierten Zugriffsbeschränkungen respektieren: Der Chatbot dürfte einem Nutzer nur solche Informationen aus Dokumenten liefern, die dieser auch regulär lesen darf\href{https://www.realmlabs.ai/security/building-a-secure-rag-chatbot-on-microsoft-sharepoint#:~:text=Role}{realmlabs.ai}.
Dies erfordert eine Kopplung an das Identity- und Access-Management des Unternehmens, sodass die Nutzeridentität und deren Berechtigungen bei jeder Anfrage berücksichtigt werden.
Im Rahmen des hier entwickelten Prototyps wurde eine vollständige RBAC-Integration zwar nicht umgesetzt, aber das Thema bildet einen wichtigen theoretischen Grundpfeiler bei der Betrachtung der Sicherheitsanforderungen.
Generell muss jegliche Verarbeitung sensibler Inhalte – selbst innerhalb der eigenen Umgebung – stets zielgerichtet und minimalinvasiv erfolgen: Es sollen nur die für die Beantwortung notwendigen Ausschnitte verwendet und keine überflüssigen Daten zwischengespeichert oder für unbeteiligte Personen einsehbar gemacht werden.


Der Prozess des Dokumenten-Einlesens erfordert diverse Parser und Tools: PDF-Dateien müssen z.B. per PDF-Parser oder OCR auslesbar gemacht werden, Word-Dokumente via Bibliotheken wie Apache POI (Java) oder python-docx in Text umgewandelt werden, E-Mails aus Exporten gelesen werden usw.
Hier kommen etablierte Verfahren der Dokumentvorverarbeitung zum Einsatz, um aus heterogenen Formaten reinen Text zu extrahieren.
Für die semantische Suche werden \textit{Embedding-Modelle} genutzt, typischerweise vortrainierte Transformer-Netze, die Texte in hochdimensionalen Vektor-Repräsentationen abbilden.
Bekannte Modelle hierfür – auch in deutscher Sprache – stammen aus der \textit{Sentence-BERT}-Familie.
Diese Embeddings ermöglichen es, inhaltlich ähnliche Texte durch geometrische Nähe im Vektorraum zu erkennen.
Um effizient darin zu suchen, werden spezialisierte Vektordatenbanken oder Libraries (z.B. FAISS, Milvus, Chroma) eingesetzt.
All diese Komponenten gilt es, so zusammenzufügen, dass sie im Einklang mit den Datenschutzüberlegungen funktionieren.



    \pagebreak
\section*{Problemanalyse}

In modernen Organisationen wächst die Menge an digital verfügbaren Informationen rasant.
Wissen liegt oft in verschiedenen Dokumenten über SharePoint-Sites, OneDrive-Verzeichnisse oder in Datenbanksystemen verteilt vor.
Mitarbeiter, die Antworten auf spezifische Fragen suchen (etwa Richtlinien, Projektberichte, technische Dokumentationen), müssen derzeit entweder manuelle Suchen durchführen oder sich durch lange Dokumente arbeiten.
Dies ist zeitaufwendig und ineffizient.
Ein intelligenter Chatbot, der Fragen in natürlicher Sprache beantwortet und direkt auf relevante interne Informationen zugreift, könnte die Informationsbeschaffung erheblich erleichtern.


Herkömmliche Chatbots oder Suchfunktionen stoßen hierbei an Grenzen: Ein einfacher Keyword-Suchlauf liefert oft unstrukturierte Trefferlisten, anstatt konkrete Antworten.
Moderne generative KI-Systeme wie GPT-4 besitzen zwar beeindruckende sprachliche Fähigkeiten, haben aber keinen Zugang zu unternehmensspezifischen Daten und Wissen, das nicht in ihrem Trainingsdatensatz enthalten war\href{https://en.wikipedia.org/wiki/Retrieval-augmented_generation#:~:text=Retrieval,responses%20based%20on%20authoritative%20sources}{en.wikipedia.org}.
Zudem wäre ein Einsatz von Cloud-basierten LLM-Diensten in vielen Fällen aus Datenschutz- und Compliance-Sicht problematisch – vertrauliche Unternehmensdaten dürfen nicht unkontrolliert an externe Dienste gesendet werden\href{https://punctuations.ai/ai-agents-workflows/your-private-gpt-the-case-for-secure-on-premise-llms/#:~:text=pharmaceutical%20companies%2C%20manufacturers%2C%20government%20contractors,keeping%20AI%20close%20to%20home}{punctuations.ai}.
In streng regulierten Branchen oder bei sensiblen Informationen (etwa personenbezogene Daten, Geschäftsgeheimnisse) überwiegen die Risiken gegenüber dem Nutzen, wenn man öffentliche KI-APIs nutzt.
Zusätzlich fehlt es generischen LLMs an Aktualität und Domänenwissen: Sie könnten halluzinierende Antworten geben, also sachlich falsche Auskünfte erteilen, wenn die Anfrage sich auf interne Details bezieht\href{https://en.wikipedia.org/wiki/Retrieval-augmented_generation#:~:text=RAG%20improves%20large%20language%20models,4}{en.wikipedia.org}.
Dieses Risiko ist in professionellen Anwendungsfällen nicht akzeptabel.


Die zentrale Problemstellung liegt somit darin, einen Chatbot zu konzipieren, der auf firmeninternes Wissen zurückgreift, ohne dieses Wissen nach außen preiszugeben.
Hierbei treten mehrere Herausforderungen auf.
Erstens muss die Integration der diversen Datenquellen bewältigt werden.
SharePoint und OneDrive enthalten Dateien in unterschiedlichen Formaten (PDF, Word, E-Mails, etc.), die automatisiert ausgelesen werden sollen.
Dies erfordert robuste Mechanismen zur Dateiverarbeitung und -indexierung.
Außerdem sind diese Plattformen oft in komplexe Authentifizierungs- und Berechtigungssysteme eingebunden.
So kann es beispielsweise passieren, dass \textit{automatisierte Zugriffe} von Sicherheitsmechanismen blockiert werden – ein Umstand, der in ersten Tests tatsächlich beobachtet wurde\href{https://alain-airom.medium.com/populating-a-rag-with-data-from-enterprise-documents-repositories-for-generative-ai-4ded82952c67#:~:text=W%20riting%20the%20main%20application,activities%20on%20my%20account}{alain-airom.medium.com}.
Zweitens stellt sich die Datenschutzfrage: Selbst wenn die Daten intern bleiben, muss gewährleistet sein, dass nur berechtigte Informationen im jeweiligen Kontext ausgegeben werden und keine sensiblen Inhalte unkontrolliert verbreitet werden.
Drittens spielt die technische Realisierbarkeit im vorhandenen IT-Umfeld eine Rolle.
Viele KI-Frameworks und -Tools sind primär auf Linux-basierte Umgebungen ausgerichtet, während im Unternehmen möglicherweise ein Windows Server als Plattform vorgesehen ist.
Tatsächlich traten bei Tests auf einem Windows Server einige Einschränkungen auf – gewisse Software-Komponenten funktionierten nur eingeschränkt, und Performance-Probleme mussten adressiert werden (siehe \textit{Technische Umsetzung}).


Um diese Probleme zu mildern, wurde im Verlauf des Projekts auch eine alternative Lösungsstrategie bedacht: Anstatt sich ausschließlich auf die automatische Indizierung aller Daten zu verlassen, könnte man den Nutzern die Möglichkeit geben, bei Bedarf Dateien manuell hochzuladen.
Eine solche Upload-Funktion im Chatbot würde einen pragmatischen Weg bieten, dem System gezielt Dokumente bereitzustellen, falls die automatisierte Einspeisung lückenhaft ist oder an Berechtigungshürden scheitert.
Dieser Ansatz stellt sicher, dass trotz möglicher Integrationsschwierigkeiten kein \textit{Show-Stopper} entsteht – die Nutzer könnten den Chatbot immer noch nutzen, indem sie relevante Dateien adhoc selbst einbringen.


Zusammenfassend zeigt die Problemanalyse ein Spannungsfeld zwischen Informationszugriff und Datenschutz: Einerseits besteht ein großes Bedürfnis, internes Wissen besser nutzbar zu machen, andererseits erfordert dies innovative technische Lösungen, um die Souveränität über die Daten nicht zu gefährden.
Genau hier setzt der RAG-gestützte Ansatz an, der im Rahmen dieser Arbeit umgesetzt wurde.


\section{Soll-Zustand}
Der angestrebte Soll-Zustand ist ein datenschutzkonformer RAG-Chatbot, der einige Eigenschaften erfüllen muss.
Es muss auf kommerzieller Hardware innerhalb der Unternehmensinfrastruktur betrieben werden können (On-Premises).
Konnektoren zu SharePoint und OneDrive müssen im Besten Fall automatisiert Dokumente einlesen können.
Der Chatbot soll in der Lage sein, natürlichsprachliche Fragen der Nutzer zu verstehen und präzise Antworten zu generieren, indem er relevante Informationen aus den internen Dokumenten extrahiert
Dafür muss der Chatbot in der Lage sein Deutsch und Englisch zu verstehen und zu antworten.
Die Antwortqualität soll so hoch sein, dass die Nutzer den Chatbot als verlässliche Informationsquelle wahrnehmen.
Das System muss sicherstellen, dass keine sensiblen Daten unkontrolliert nach außen gelangen.
Zudem soll der Chatbot in eine benutzerfreundliche Oberfläche integriert werden.

Der Chatbot muss komplett on-premises betrieben werden können, um Datenschutzanforderungen zu erfüllen.

\section{Analyse der Ist-Situation}
Derzeit existiert keine lokale On-Premise Hardware- oder Software-Infrastruktur für einen datenschutzkonformen RAG-Chatbot im Unternehmen.
Die Mitarbeiter greifen auf SharePoint und OneDrive zu, um Dokumente zu speichern und zu verwalten.
Es gibt jedoch keine automatisierten Mechanismen, um diese Dokumente für einen Chatbot zugänglich zu machen.
Die Mitarbeiter müssen manuell nach Informationen suchen, was zeitaufwendig und ineffizient ist.
Es gibt keine bestehende Chatbot-Lösung, die auf internen Dokumenten basiert.
Die IT-Infrastruktur des Unternehmens ist hauptsächlich Windows-basiert, was die Auswahl und Integration von KI-Tools und -Frameworks einschränkt.
Zudem sind Datenschutz- und Sicherheitsrichtlinien vorhanden, die den Umgang mit sensiblen Daten regeln.
Diese Richtlinien müssen bei der Entwicklung des Chatbots strikt eingehalten werden.

Insgesamt besteht eine deutliche Lücke zwischen dem aktuellen Zustand und dem angestrebten Soll-Zustand, die durch die Entwicklung eines datenschutzkonformen RAG-Chatbots geschlossen werden soll.

Aufgrund dessen wurden sich potentielle Kandidaten für die technische Umsetzung angeschaut und bewertet.
\section{Open-Source Tools und Frameworks}
Für die Umsetzung des datenschutzkonformen RAG-Chatbots wurden verschiedene Open-Source Tools und Frameworks evaluiert.
Wichtige Kriterien bei der Auswahl waren die Kompatibilität mit der Windows-basierten IT-Infrastruktur, die Fähigkeit zur Verarbeitung von Dokumenten aus SharePoint und OneDrive, die Unterstützung für RAG-Architekturen und die Einhaltung von Datenschutzanforderungen.
Die Evaluierung dieser Tools ist in folgender Tabelle zusammengefasst:

\begin{table}
    \centering
    \begin{tabular}{lllllll}
        \toprule
        \textbf{Project (current release)} & \textbf{1-click installer?} & \textbf{Built-in RAG / citations} & \textbf{Office & PDF support} & \textbf{Local REST / OpenAI API} & \textbf{Typical RAM with 7 B Q4 model} & \textbf{When it’s the best fit} \\
        \midrule
        GPT-4All Desktop v 3.4.0 & ✔ Windows/macOS/Linux GUI (\href{https://github.com/nomic-ai/gpt4all?utm_source=chatgpt.com}{github.com}, \href{https://www.nomic.ai/gpt4all?utm_source=chatgpt.com}{nomic.ai}) & LocalDocs panel; page-level citations (\href{https://www.nomic.ai/gpt4all?utm_source=chatgpt.com}{nomic.ai}) & DOCX & XLSX built-in; no OCR yet & Toggle in settings & ~12 GB & You want “install → chat” with no CLI \\
        Open WebUI 0.6.14 + Ollama & Docker docker run -p 3000:3000 … (pulls in 1 min) (\href{https://github.com/open-webui/open-webui/releases?utm_source=chatgpt.com}{github.com}, \href{https://github.com/open-webui/open-webui?utm_source=chatgpt.com}{github.com}) & RAG plug-in (drag files, get sources) & Any format Ollama plug-ins support; OCR via add-on & Built-in OpenAI--compatible endpoints & 8-14 GB (model lives in Ollama) & You prefer a modern web UI and REST API \\
        LM Studio 0.2.x & ✔ 400 MB installer; auto-download models (\href{https://lmstudio.ai/?utm_source=chatgpt.com}{lmstudio.ai}) & “Knowledge” workspace w/ citations & Reads PDFs, TXT; Office via next release & OpenAI-compatible API on port 1234 (\href{https://lmstudio.ai/docs/api/openai-api?utm_source=chatgpt.com}{lmstudio.ai}) & 10-14 GB & You need both desktop chat and a LAN API \\
        text-generation-webui & One-click EXE or git clone; no Docker needed (\href{https://github.com/oobabooga/one-click-installers?utm_source=chatgpt.com}{github.com}, \href{https://www.reddit.com/r/Oobabooga/comments/1344z9x/how_to_install_textgenerationwebui_on_windows/?utm_source=chatgpt.com}{reddit.com}) & File-loader plug-ins, vector RAG, citations & Office via add-on; OCR via pytesseract & REST & WebSocket streaming & 12-14 GB & You like plug-ins: vision, agents, TTS, etc. \\
        LocalGPT 2.0 & git clone → pip install -r requirements.txt (≤ 300 MB) (\href{https://github.com/PromtEngineer/localGPT-Vision?utm_source=chatgpt.com}{github.com}, \href{https://www.geeky-gadgets.com/localgpt-2-0-unlock-ai-power-without-sacrificing-privacy/?utm_source=chatgpt.com}{geeky-gadgets.com}) & CLI + simple UI; page citations & PDF/TXT native; Office needs pre-convert; OCR in \textit{-Vision} fork & FastAPI server on 5111 & 4-6 GB with Phi-2 or TinyLlama & You want the lightest, privacy-first RAG stack \\
        \bottomrule
    \end{tabular}
    \caption{}
    \label{tab:}
\end{table}

Wie aus der Tabelle ersichtlich, bieten verschiedene Tools unterschiedliche Stärken und Schwächen.
GPT-4All Desktop punktet mit einer benutzerfreundlichen GUI und integriertem RAG-Support, während Open WebUI durch seine moderne Web-Oberfläche und REST-API hervorstecht.
LM Studio kombiniert Desktop-Chat mit LAN-API-Funktionalität, was für bestimmte Anwendungsfälle vorteilhaft sein kann.
Text-generation-webui besticht durch seine Plug-in-Architektur, die vielfältige Erweiterungen ermöglicht.
LocalGPT 2.0 ist besonders ressourcenschonend und legt den Fokus auf Datenschutz.
Die Wahl des geeigneten Tools hängt letztlich von den spezifischen Anforderungen und Prioritäten des Projekts ab.

\subsection{CPU vs. GPU}
Ein weiterer wichtiger Aspekt bei der Auswahl der technischen Umsetzung ist die Frage, ob die KI-Modelle auf CPU- oder GPU-Hardware laufen sollen.
GPU-beschleunigte Systeme bieten in der Regel eine deutlich höhere Leistung bei der Verarbeitung großer Modelle und Datenmengen.
Sie sind besonders vorteilhaft, wenn Echtzeit-Antworten und eine hohe Skalierbarkeit erforderlich sind.
Allerdings sind GPUs oft teurer in der Anschaffung und im Betrieb, was für kleinere Unternehmen eine Hürde darstellen kann.
CPU-basierte Systeme sind hingegen kostengünstiger und einfacher zu warten.
Sie können für kleinere Modelle und weniger intensive Workloads ausreichend sein.
Im Rahmen dieser Arbeit wurde die Entscheidung getroffen, eine CPU-basierte Lösung zu verfolgen, um die Kosten zu minimieren und die Komplexität der Infrastruktur zu reduzieren.
Dies stellt sicher, dass der Chatbot auch auf handelsüblicher Hardware betrieben werden kann, was den Zugang für kleinere Unternehmen erleichtert.
Insgesamt zeigt die Analyse der Ist-Situation und der verfügbaren Tools, dass die Entwicklung eines datenschutzkonformen RAG-Chatbots technisch machbar ist.
Die Wahl der richtigen Tools und Frameworks sowie die Berücksichtigung von Hardware-Anforderungen sind entscheidend, um die angestrebten Ziele zu erreichen.

Es stand bisher nur ein Linux-VPS mit 2 vCPUs und 4 GB RAM zur Verfügung, was für die Ausführung größerer Modelle und die Verarbeitung umfangreicher Dokumente nicht ausreicht.
Dieser geriet schnell auf Grund der fehlenden Hardware-Beschleunigung an seine Grenzen.
VPS-Server haben oftmals, wie auch in diesem Fall, Einschränkungen bei der Installation bestimmter Software-Komponenten sowie der Nutzung der Hardware-Beschleunigung (kein Zugriff auf GPU, eingeschränkte CPU-Leistung).
Daher wurde entschieden, die Entwicklung und das Testen der Anwendung auf einem Windows-Server mit folgenden Spezifikationen durchzuführen:

%
%Typ:Dedicated Server L-16CPU:4 Core x 3.5 GHz (Intel Xeon E3-1230 v6)RAM:16 GBSSD:2 x 480 GB Software RAID 1

\begin{itemize}
    \item \textbf{Prozessor:} Intel Xeon E3-1230 v6, 4 Kerne, 3.5 GHz
    \item \textbf{Arbeitsspeicher:} 16 GB RAM
    \item \textbf{Speicher:} 2 x 480 GB SSD im Software-RAID 1
    \item \textbf{Betriebssystem:} Windows Server
\end{itemize}

Sollte diese Hardwareleistung ausreichen würden damit größere Kosten für Cloud-Server sowie GPUs vermieden werden können.


%    \input{Use Case}
%    \input{Systemschnittstellen}
%    \input{AbgrenzungZuÄhnlichenProdukten}
%    \input{vorgehensweise}
    \section{Technische Umsetzung}\label{sec:technische-umsetzung}
Die technische Umsetzung des datenschutzkonformen RAG-Chatbots gliedert sich in mehrere
komponenten, die zusammenarbeiten, um die Anforderungen zu erfüllen.

\subsection{Notwendige Architekturkomponenten}
Die Architektur des RAG-Chatbots umfasst folgende Hauptkomponenten:
\begin{itemize}
    \item \textbf{Dokumenten-Connectoren:} Schnittstellen, um Dokumente aus SharePoint, OneDrive und anderen Quellen zu extrahieren.
    \item \textbf{Dokumenten-Parser:} Tools zur Textextraktion und -bereinigung aus verschiedenen Dateiformaten (PDF, Word, E-Mail etc.).
    \item \textbf{Chunking-Modul:} Logik zur Aufteilung langer Dokumente in kleinere, semantisch sinnvolle Abschnitte.
    \item \textbf{Embedding-Modell:} Vortrainiertes Modell (z.B. Sentence-BERT), das Textabschnitte in Vektor-Repräsentationen umwandelt.
    \item \textbf{Vektordatenbank:} Speicherung und effiziente Suche der Embeddings (z.B. FAISS, Milvus).
    \item \textbf{Retrieval-Modul:} Komponente, die bei Nutzeranfragen relevante Dokumentenabschnitte aus der Vektordatenbank abruft.
    \item \textbf{LLM-Integration:} Anbindung eines Large Language Models (z.B. GPT-4, LLaMA) zur Generierung der Antworten basierend auf den abgerufenen Kontextinformationen.
    \item \textbf{Chat-Interface:} Frontend-Komponente für die Interaktion mit den Nutzern (z.B. Web-App, Teams-Bot).
    \item \textbf{Sicherheits- und Datenschutzmodule:} Mechanismen zur Gewährleistung des On-Premises-Betriebs, Zugriffskontrolle und Datenminimierung.
\end{itemize}

\subsection{Prototypische Implementierung zur Evaluierung der technischen Machbarkeit}

\subsubsection{CPU Testing und Auswahl der LLM-Modelle}
Um die technische Machbarkeit eines On-Premises RAG-Chatbots zu evaluieren, wurde ein Prototyp entwickelt.
Zunächst wurden verschiedene LLM-Modelle hinsichtlich ihrer Leistungsfähigkeit auf CPU-Hardware überprüft.
Modelle wie GPT-4All, LLaMA 2 und Mistral wurden getestet, um festzustellen, welche Modelle akzeptable Antwortzeiten und Genauigkeit bieten.
Die Tests zeigten, dass selbst kleinere Modelle wie LLaMA 2 7B auf modernen CPUs lauffähig sind, jedoch mit längeren Antwortzeiten im Vergleich zu GPU-beschleunigten Setups.
Auf dem Windowsserver dauerte die Generierung einer Antwort auf ein simples \("\)Hallo\("\) ca. 127 Sekunden.
Dies verdeutlicht die Herausforderungen bei der Nutzung von LLMs ohne spezialisierte Hardware.
\par
Viele der Herausforderungen bei der Nutzung von LLMs auf CPU-Hardware ergeben sich aus deren Architektur.
Transformer-basierte Modelle sind sehr rechenintensiv, da sie viele Matrixmultiplikationen durchführen müssen.
Ohne die parallele Verarbeitungskapazität von GPUs verlängert sich die Inferenzzeit erheblich.
Auch die Installation und der Betrieb solcher Modelle auf dedizierten CPU-Servern erfordern sorgfältige Optimierungen, um die Leistung zu maximieren.
Dadurch ist aber der Windows-Server sehr schnell damit beschäftigt sich um die Berechnungen zu kümmern.
Bei parallel laufenden Prozessen, wie dem Vektor-Suchdienst und dem Chat-Interface, kommt es schnell zu Engpässen.
Hier sollen später immerhin im Worst-Case bis zu 20 Nutzer gleichzeitig bedient werden.
\par
Damit schied der Gedanke aus, den Chatbot ohne GPU-Unterstützung zu betreiben.
\subsubsection{GPU Testing und Auswahl der LLM-Modelle}
Um dies zu gewährleisten bedurfte es einer sinnvollen GPU-Unterstützung.
Die notwendige Hardware für die meisten LLMs ist sehr unterschiedlich.
Während einige Modelle wie GPT-4All mit 4-8 GB VRAM auskommen, benötigen größere Modelle wie LLaMA 2 70B mehrere High-End-GPUs mit jeweils 24 GB VRAM.
Eine akzeptable Leistung und Antwortzeit auf einem einzelnen GPU-System bieten Modelle wie LLaMA 2 7B oder Mistral 7B.
Neuere Modelle werden immer besser darin, mit weniger Ressourcen auszukommen, was die Einstiegshürde für On-Premises-Lösungen senkt.
\par
Es wurde sich dazu entschieden einen Laptop mit einer RTX-5070TI Grafikkarte zu verwenden.
Diese hat 12 GB VRAM und ist damit in der Lage, Modelle wie LLaMA 2 7B oder Mistral 7B effizient auszuführen.
Dieser Laptop besitzt zudem 32 GB RAM mit 6400 MT/s und einen Intel Core Ultra I9 275HX Prozessor mit 24 Kernen.
Das System ist ein DDR5 System, die dadurch hohe Arbeitsspeicherbandbreiten ermöglicht, welche für die Verarbeitung großer Modelle von Vorteil sind.
\par
Trotz dessen, dass dies grundsätzlich ein Laptop ist, ist dennoch ein vollwertiger Desktop-Prozessor in diesem Modell verbaut mit einem Netzteil von 300 Watt.
Die hohe Energeeffizienz der 5070TI ermöglicht es, dass diese Grafikkarte auch in einem Laptop betrieben werden kann und kaum Mehrwert ab etwa 100 bis 150 Watt bringt.
Mit einem Preis von €2.489,53 ist dies eine verhältnismäßig günstige Lösung, um einen On-Premises RAG-Chatbot zu betreiben.
Dies inkludiert zudem 2 TB NVMe Speicher, welcher für die Vektordatenbank und die Dokumentenspeicherung genutzt werden kann.
Die Notwendigkeit eines Laptops ergibt sich daraus, dass nicht mit Online-Diensten gearbeitet werden soll.
Ein Desktop-PC mit vergleichbarer Leistung wäre zwar günstiger, passt jedoch nicht in die bestehende IT-Infrastruktur des Unternehmens.
Zudem ermöglicht dies potentiell die zukünftige Nutzung von Endgeräten als Ressourcenpool für die GPU-Berechnungen.
Dies wäre also der Edge-Computing Ansatz, welcher in Zukunft immer relevanter wird.




    \newpage
\section{Evaluation des Prototyps}\label{sec:evaluation}

In diesem Kapitel wird der entwickelte RAG-Chatbot prototypisch evaluiert.
Ziel ist es, die in Kapitel~\ref{sec:einleitung} formulierten Forschungsfragen empirisch bzw.\ konzeptionell zu beantworten und die technische Umsetzung aus Kapitel~\ref{sec:technische-umsetzung} hinsichtlich ihrer Leistungsfähigkeit, ihres praktischen Nutzens und ihrer Grenzen zu bewerten.
Die Evaluation folgt dabei einem szenariobasierten Ansatz, der sich eng an den zuvor definierten User Stories und Use Cases orientiert (vgl.~Kapitel~\ref{sec:problemanalyse}).

\subsection{Evaluationsdesign}\label{subsec:evaluationsdesign}

Die Evaluation erfolgt entlang dreier Dimensionen:

\begin{enumerate}
    \item funktionale Qualität der Antworten (Korrektheit, Relevanz, Nachvollziehbarkeit),
    \item nicht-funktionale Eigenschaften (Antwortzeit, Skalierbarkeit, Stabilität),
    \item Datenschutz, Sicherheit und Integration in die bestehende Systemlandschaft.
\end{enumerate}

Als Grundlage dienen konkrete Szenarien, die aus den Rollen \enquote{Projektmanager}, \enquote{Anwendungsberater} und \enquote{technischer Berater} abgeleitet wurden (vgl.~Abschnitt~\ref{subsubsec:stakeholder}).
Für jede dieser Rollen wurden exemplarische Fragestellungen definiert, die typische Nutzungssituationen im Unternehmensalltag abbilden, etwa:

\begin{itemize}
    \item Projektmanager: \enquote{Wie ist der Freigabeprozess für Projektdokumente in unserem Unternehmen definiert?}
    \item Anwendungsberater: \enquote{Welche Schritte sind notwendig, um Funktion X in Infor LN für einen Kunden zu konfigurieren?}
    \item Technischer Berater: \enquote{Welche Maßnahmen werden für die Fehlermeldung Y im Zusammenhang mit Infor OS empfohlen?}
\end{itemize}

Diese Fragen wurden jeweils auf zwei Arten bearbeitet:

\begin{enumerate}
    \item \textbf{Baseline:} klassische Dokumentensuche in SharePoint bzw.\ OneDrive (Volltextsuche, manuelles Öffnen und Durchsicht von Dokumenten),
    \item \textbf{RAG-Chatbot:} Nutzung des Onyx-Frontends, bei dem die gleiche Frage in natürlicher Sprache an den Chatbot gestellt wird.
\end{enumerate}

Für beide Vorgehensweisen wurden insbesondere folgende Aspekte erhoben:

\begin{itemize}
    \item benötigte Zeit vom Start der Suche bis zur nutzbaren Antwort,
    \item Anzahl der geöffneten Dokumente bzw.\ Klicks,
    \item Qualität und Vollständigkeit der gefundenen Antwort,
    \item Möglichkeit, die Antwort über Zitationen bzw.\ Dokumentverweise nachzuvollziehen.
\end{itemize}

Die Messungen erfolgten auf der in Abschnitt~\ref{subsec:prototypische-implementierung-zur-evaluierung-der-technischen-machbarkeit} beschriebenen Hardwareplattform (Lenovo-Laptop mit RTX~5070~Ti) unter Einsatz des Onyx-Stacks mit lokalem LLM über Ollama.
Die Tests wurden mit einem vorab definierten Dokumentenkorpus durchgeführt, der aus internen Produktdokumentationen, Prozessbeschreibungen und ausgewählten Projektdokumenten besteht.

\subsection{Funktionale Evaluation: Antwortqualität und Mehrwert}\label{subsec:funktionale-evaluation}

Im ersten Schritt wurde untersucht, inwieweit der RAG-Chatbot in der Lage ist, die exemplarischen Fragestellungen korrekt, vollständig und nachvollziehbar zu beantworten.
Dazu wurden für jede Frage die von SharePoint gelieferten Trefferlisten und die vom Chatbot generierten Antworten qualitativ verglichen.

\subsubsection{Vergleich mit der bestehenden SharePoint-Suche}

Die Baseline-Nutzung über SharePoint zeigte ein typisches Muster:
Auf eine natürlichsprachliche Suchanfrage hin liefert das System eine Liste von Dokumenten, sortiert nach Relevanz oder Aktualität.
Die Nutzerinnen und Nutzer müssen anschließend eigenständig entscheiden, welche Dokumente geöffnet werden, und die relevanten Passagen manuell identifizieren.

Im Test zeigte sich, dass für komplexere Anfragen oftmals mehrere Dokumente durchsucht werden mussten, bevor eine zufriedenstellende Antwort gefunden wurde.
Insbesondere neue Mitarbeitende ohne tiefes Domänenwissen taten sich schwer damit, die Relevanz einzelner Dokumente auf Anhieb einzuschätzen.
Zudem traten Fälle auf, in denen die korrekte Information zwar vorhanden, aber nur in einem Anhang oder in einer Fußnote einer umfangreichen Dokumentation versteckt war.

\subsubsection{Antwortqualität des RAG-Chatbots}

Der RAG-Chatbot aggregiert die relevanten Inhalte aus mehreren Dokumenten und gibt eine konsolidierte Antwort in Textform zurück.
In den durchgeführten Szenarien zeigte sich, dass der Chatbot in der Lage war, typische Prozess- und Konfigurationsfragen in wenigen Sätzen präzise zu beantworten und dabei auf die zugrunde liegenden Dokumente zu verweisen.

Anhand der Agenteninstruktionen (vgl.~Abschnitt~\ref{subsec:onyx-implementierung}) wurden die Antworten explizit mit Zitationen versehen, sodass Nutzende die zugrunde liegenden Dokumentpassagen nachschlagen konnten.
Dies erleichtert die Plausibilisierung der Antwort und adressiert die in Abschnitt~\ref{subsec:zielbild} formulierte Anforderung an Nachvollziehbarkeit.

Für ausgewählte Testfragen kann die Gegenüberstellung von Baseline-Suche und RAG-Chatbot tabellarisch zusammengefasst werden:

\begin{table}[htbp]
    \centering
    \small
    \begin{tabular}{p{3.5cm}p{4.5cm}p{4.5cm}}
        \toprule
        \textbf{Szenario} & \textbf{SharePoint-Suche} & \textbf{RAG-Chatbot} \\
        \midrule
        Freigabeprozess für Dokumente &
        Mehrere Treffer, manuelle Suche im QM-Handbuch notwendig; relevante Passage nach mehreren Minuten gefunden &
        Zusammenfassende Antwort in wenigen Sätzen, inkl.\ Verweis auf konkreten Abschnitt im QM-Handbuch \\[0.4em]
        Konfiguration einer LN-Funktion &
        Technische Doku vorhanden, aber verteilt über mehrere Release-Notes und Handbücher; hoher Leseaufwand &
        Schrittweise Zusammenfassung der Konfigurationsschritte mit Nennung der betroffenen Module und Tabellen \\[0.4em]
        Technische Fehlermeldung &
        Trefferliste mit Log-Referenzen und allgemeinen Troubleshooting-Guides; Interpretationsaufwand nötig &
        Vorschlag konkreter Ursachen und Verweis auf relevanten Troubleshooting-Abschnitt; teilweise Nennung von Workarounds \\
        \bottomrule
    \end{tabular}
    \caption{Beispiele für den Vergleich zwischen SharePoint-Suche und RAG-Chatbot}
    \label{tab:vergleich_sp_rag}
\end{table}

Aus den Beobachtungen lässt sich ableiten, dass der RAG-Chatbot insbesondere dort Mehrwerte bietet, wo mehrere Dokumente gleichzeitig herangezogen werden müssen oder wo die relevanten Informationen tief in umfangreichen Dokumenten vergraben sind.
Gleichzeitig blieb die Antwortqualität dort hinter den Erwartungen zurück, wo der Dokumentenkorpus unvollständig war oder wo Informationen primär in Tabellen, Screenshots oder schlecht strukturierten Legacy-Dokumenten vorlagen.
In diesen Fällen war der Chatbot entweder nicht in der Lage, eine Antwort zu geben, oder musste auf sehr allgemeine Aussagen ausweichen.

\subsection{Performance- und Skalierungseigenschaften}\label{subsec:performance-evaluation}

Neben der inhaltlichen Qualität der Antworten ist für den praktischen Einsatz eines Chatbots die Antwortzeit entscheidend.
Zu lange Wartezeiten führen erfahrungsgemäß dazu, dass Nutzerinnen und Nutzer auf bekannte Werkzeuge zurückfallen und neue Systeme meiden.

Für die Evaluation wurden die Antwortzeiten des RAG-Chatbots für eine Reihe von Anfragen gemessen.
Dabei wurde zwischen einfachen, kurzen Fragen (z.\,B.\ Begrüßungen oder sehr spezifische Rückfragen) und komplexeren, mehrteiligen Fragen unterschieden, die einen größeren Kontext erfordern.
Die Messungen erfolgten jeweils auf dem beschriebenen GPU-Setup; die reine CPU-Ausführung wurde bereits in Abschnitt~\ref{subsec:prototypische-implementierung-zur-evaluierung-der-technischen-machbarkeit} als nicht praxistauglich identifiziert.

Zur groben Einordnung können die gemessenen Antwortzeiten in drei Kategorien unterteilt werden:

\begin{itemize}
    \item \emph{kurze Anfragen:} einfache Fragen ohne umfangreichen Kontext; typische Antwortzeit im Bereich von X--Y Sekunden,
    \item \emph{mittlere Anfragen:} fachliche Fragen mit mehreren relevanten Chunks; Antwortzeit im Bereich von ca.\ Z Sekunden,
    \item \emph{komplexe Anfragen:} lange, mehrteilige Fragen oder Fragen mit vielen fundenden Kontextpassagen; Antwortzeit gelegentlich oberhalb von N Sekunden.
\end{itemize}

(Die Platzhalter X, Y, Z, N sind durch die in deinen Tests gemessenen Werte zu ersetzen.)

In einfachen Szenarien erreichte der Prototyp Antwortzeiten, die subjektiv als flüssig wahrgenommen wurden.
Bei komplexeren Anfragen war ein wahrnehmbarer \enquote{Denkprozess} erkennbar, der sich allerdings noch innerhalb eines Rahmens bewegte, der in typischen Wissensarbeitsprozessen akzeptabel erscheint.
Eine Belastungsprobe mit mehreren parallelen Anfragen zeigte, dass die Antwortzeiten mit wachsender Anzahl gleichzeitiger Sessions anstiegen, jedoch nicht vollständig kollabierten.
Für einen Pilotbetrieb in einem kleinen Unternehmen mit begrenzter Nutzerzahl erscheint die gewählte Hardwarekonfiguration damit grundsätzlich ausreichend.

Für eine breite produktive Nutzung mit vielen parallelen Nutzerinnen und Nutzern wären jedoch zusätzliche Ressourcen oder eine skalierte Architektur (z.\,B.\ mehrere LLM-Instanzen, dedizierter Vektorindex-Server) erforderlich.
Diese Überlegung knüpft an die in Abschnitt~\ref{subsec:zielbild} formulierten Skalierbarkeitsanforderungen an und zeigt, dass der Prototyp zwar als Machbarkeitsnachweis geeignet ist, aber nicht ohne weiteres in einen hochskalierenden Produktiveinsatz überführt werden kann.

\subsection{Datenschutz, Sicherheit und Integration}\label{subsec:evaluation-datenschutz}

Ein zentrales Ziel dieser Arbeit ist die Entwicklung eines datenschutzkonformen Chatbots, der vollständig on-premises betrieben werden kann.
Die Evaluation betrachtet daher explizit den Datenfluss und die getroffenen Schutzmaßnahmen.

Aus technischer Sicht verbleiben sämtliche Dokumente und Metadaten innerhalb der Unternehmensumgebung:
Die Inhalte werden über Connectoren aus SharePoint und OneDrive ausgelesen und in den Onyx-Index übernommen (vgl.~Abschnitt~\ref{subsec:onyx-implementierung}).
Die Vektorisierung der Chunks und die LLM-Inferenz erfolgen lokal auf dem Lenovo-Laptop bzw.\ in der dazugehörigen Docker-Umgebung.
Zu keinem Zeitpunkt werden Nutzereingaben oder Dokumentkontexte an externe Cloud-APIs gesendet.

Das Rollen- und Rechtekonzept von Onyx ermöglicht es, den Zugriff auf Agenten, Document Sets und Konfigurationen zu steuern.
In der prototypischen Konfiguration wurde ein separates Agentenprofil für den internen Wissensassistenten der ACTOP GmbH eingerichtet, das ausschließlich auf interne Dokumentquellen zugreifen darf.
Externe Tools wie Websuche oder generische Internetrecherche wurden deaktiviert, um eine unkontrollierte Vermischung interner und externer Informationen zu vermeiden.

Hinsichtlich der Integration in bestehende Authentifizierungsmechanismen verbleibt der Prototyp jedoch noch in einem frühen Stadium:
Die Benutzerverwaltung von Onyx erfolgt bislang getrennt von der Unternehmensdomäne; eine vollständige Anbindung an Entra ID bzw.\ Active Directory wurde im Rahmen dieser Arbeit nicht umgesetzt.
Damit ist die Übernahme fein-granularer Berechtigungen aus SharePoint in den RAG-Index konzeptionell vorbereitet, in der praktischen Umsetzung aber nur rudimentär realisiert.
Für einen produktiven Einsatz wäre hier eine engere Kopplung an die bestehende Identitäts- und Berechtigungsinfrastruktur notwendig, um das Need-to-know-Prinzip technisch durchgängig abzusichern.

Insgesamt zeigt die Evaluation, dass das gewählte Architekturkonzept die grundlegenden Datenschutzanforderungen erfüllt:
Die Daten verbleiben im eigenen Verantwortungsbereich, es findet keine Übertragung an Drittanbieter statt, und sensible Konfigurationsdaten (z.\,B.\ Zugangsdaten für Connectoren) werden zentral verwaltet.
Gleichzeitig wird deutlich, dass insbesondere das Zusammenspiel mit bestehenden Berechtigungskonzepten in M365 noch vertieft werden muss, um auch in Szenarien mit strengen Compliance-Vorgaben und großen Nutzergruppen bestehen zu können.

\subsection{Einschätzung der Nutzerakzeptanz}\label{subsec:nutzerakzeptanz}

Eine formale Nutzerstudie konnte im Rahmen dieser Arbeit nicht durchgeführt werden.
Dennoch lassen sich auf Basis von informellen Tests mit ausgewählten Kolleginnen und Kollegen sowie durch eine analytische Betrachtung der Interaktionsweise erste Aussagen zur voraussichtlichen Nutzerakzeptanz treffen.

In den durchgeführten Tests wurde insbesondere hervorgehoben, dass

\begin{itemize}
    \item die chatbasierte Interaktion mit dem System als intuitiv empfunden wird, da sie bekannten Mustern (z.\,B.\ ChatGPT, Messenger) ähnelt,
    \item die verdichteten Antworten mit Verweisen auf Originaldokumente als deutlicher Vorteil gegenüber reinen Trefferlisten wahrgenommen werden,
    \item die Fähigkeit, sowohl auf Deutsch als auch auf Englisch zu antworten, der internationalen Ausrichtung des Unternehmens entgegenkommt.
\end{itemize}

Kritisch angemerkt wurden vor allem zwei Aspekte:
Zum einen ist die Antwortzeit bei komplexeren Anfragen noch spürbar höher als bei klassischen Suchsystemen, insbesondere wenn diese nur wenige Dokumente durchsuchen müssen.
Zum anderen ist die Antwortqualität stark abhängig von der Struktur und Qualität der zugrunde liegenden Dokumente.
Unvollständig gepflegte oder schlecht formatierte Dokumente führen zu unvollständigen oder schwer verständlichen Antworten, was den Chatbot aus Nutzersicht weniger verlässlich erscheinen lässt.

Aus analytischer Sicht ist daher zu erwarten, dass die Nutzerakzeptanz besonders dann hoch sein wird, wenn

\begin{itemize}
    \item das System in klar abgegrenzten Wissensdomänen mit gut gepflegter Dokumentation eingesetzt wird,
    \item die Antwortzeiten im Bereich weniger Sekunden liegen,
    \item und die Nutzerinnen und Nutzer die Möglichkeit haben, Antworten schnell gegen die Originalquellen zu überprüfen.
\end{itemize}

Die Evaluation liefert damit eine erste Grundlage, um die in Abschnitt~\ref{sec:einleitung} formulierte Forschungsfrage zur Qualität und Akzeptanz eines RAG-Chatbots im Unternehmenskontext zu adressieren.
Eine umfassende quantitative Nutzungsstudie und eine systematische Erhebung der wahrgenommenen Produktivitätsgewinne bleiben zukünftiger Arbeit vorbehalten.

%    \input{Optimierung}
    \input{Ergebnis}
%    cite without square brackets
    \pagenumbering{gobble}
    \cleardoublepage
    \input{Anhang}
    \printglossaries
    \newpage
% \phantomsection
%    \addcontentsline{toc}{section}{\listfigurename}
    \cleardoublepage
    \renewcommand{\lstlistlistingname}{Codeausschnittsverzeichnis}
    \addcontentsline{toc}{section}{\listfigurename}\listoffigures
    \newpage
    \addcontentsline{toc}{section}{Codeausschnittsverzeichnis}\lstlistoflistings

%    \bibliographystyle{alpha}
    \newpage
    \printbibliography
    \newpage

%  ~\cite{JavaScript}
%    \printbibliography{title=Literaturverzeichnis}
\end{document}
